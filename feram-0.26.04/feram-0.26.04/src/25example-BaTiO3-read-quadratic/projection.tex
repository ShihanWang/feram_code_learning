\documentclass[a4paper,12pt]{jarticle}
\usepackage[dvipdfmx]{graphicx}
\usepackage{amsmath}
%\usepackage{wrapfig}
\usepackage{bm}
\usepackage{amssymb} % for \rightleftarrows
\renewcommand{\refname}{}
\setlength{\oddsidemargin}{-5.4mm}  % 25.4 - 5.4 = 20
\setlength{\topmargin}{-18mm}
\setlength{\textwidth}{95mm}       % 210 - 20*2 = 170 mm
\setlength{\textheight}{260mm}
\newcommand{\ab}{{\alpha\beta}}
\newcommand{\dab}{{\delta_\ab}}
\newcommand{\dij}{{\delta_{ij}}}
\newcommand{\Rij}{{\bm{R}_{ij}}}
\newcommand{\Rijn}{{\bm{R}_{ij}+\bm{n}}}
\newcommand{\Gunit}{{\frac{2\pi}{a_0}}}
\newcommand{\eInf}{{\epsilon_\infty}}
\newcommand{\La}{{L_\alpha}}
\begin{document}
\pagestyle{empty}
\thispagestyle{empty}


\vspace*{15cm}

\noindent$\Gamma$点で、ソフトモードは「3重に」縮退($\alpha=x,y,z$).
その$3n_\mathrm{at}=15$次元ノーマルモードベクトルを$\bm{\xi}^\Gamma_\alpha$.
\begin{equation}
  \label{eq:symmetrize}
  s_{\alpha, \beta I} (\bm{k})=
  \frac{1}{n_I}\sum_{J=1}^{n_I}\,\xi^\Gamma_{\alpha, \beta I}\,\exp( i \bm{k}\cdot \bm{r}^I_J)
\end{equation}
で「対称化した」ソフトモード$\bm{s}_\alpha(\bm{k})$を作る(規格化も必要?).
それにIFC行列を{\bf 射影}すると
\begin{equation}
  \label{eq:projection}
  \Phi_\ab (\bm{k}) = \bm{s}_\alpha^\dagger(\bm{k}) \Phi(\bm{k}) \bm{s}_\beta (\bm{k}).
\end{equation}


18日午前
ポスター
新屋 Ge:MnはGeMn4かなにかのクラスター, FeはクラスターFeはGe latticeのオンサイト

NTE ラマン散乱実験。

小口研 LiMnTi2O4, バンド計算はフェロで。電圧とLi, M, Oが抜けるのに必要なエネルを評価。

18日午後
放射光X線を用いたフォノン物性研究の現状
ダイヤモンドで170meV。分解能が数meV必要。
ブラッグ角が90度に近いと分解能を稼げる。

第一原理フォノン計算テクノロジ 東後さん
いつもどおり。

ナノ構造の熱伝導
予想を超えるものはなかった。

水滴の振動のやつ
洗剤、インクジェット

GHz, フォノニック結晶
禁制周波数/バンドギャップを持つことができる

複合材量積層構造の超音波バンドギャップと非破壊材料評価
よくわからなかった。

アミロイドーシス
超音波による結晶性アミロイド形成の加速
キャビテーションが核になるのか?
エコカイロ(酢酸ナトリウム)
結石症
鎌形赤血球
過飽和なしのアモルファス凝集

19日午前 総合講演
量子アニーリング 茶色ノートに書いた。
量子重力

19日午後
AVX2 渡辺さん 茶色ノートに書いた。

水素吸蔵のダイナミクス(折茂研佐藤豊人)
吸蔵しやすいってことは、放出しにくいということ。なかなか実用化は難しそう。
ミューオンや中性子の非弾性散乱実験で。

たんぱく質の動的構造研究とミュオン科学の融合(名工大)



超低速ミュオンJ-PARC幸田さん
43個/s


電子水素イオン相関
明石高専中西寛
Naniwa 個体中の水素やミュオンなどを量子論的に扱える



20pB21-1
KPO. よくわからん。

20pB21-2

20pB21-3
京コンピュータ。Hadamard gateってのは簡単に実装できる。
上位M bitを 2^M MPIプロセスで。
下位N bitをプロセスないのメモリで。
量子加算

20pB21-4
わかんね

20pB21-5
わかんね
なんか、スマフォのプレゼンツールを使ってた。

20pB21-6
量子多体型
ランダムな時間発展=ランダムなユニタリ
Haar random unitary 〜 random matrix theory??? O(2^2^n)
近似 Unitary t-design t=1,2,3,4,5,...
スピングラス O(t)


20pB21-7





\end{document}
